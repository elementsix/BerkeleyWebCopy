\Question{Solution for $ax \equiv b \pmod m$}

In the notes, we proved that when $\gcd(m, a) = 1$, $a$ has a unique multiplicative inverse, or equivalently $ax \equiv 1\pmod m$ has exactly one solution $x$ (modulo $m$).
This proof also implies that when $\gcd(m, a) = 1$, there is a unique solution to $ax \equiv b\pmod m$, where $x$ is the unknown variable.

Now consider the equation $ax \equiv b\pmod m$, when $\gcd(m, a)>1$.

\begin{Parts}

	\Part Let $\gcd(m, a) = d$.
	Prove that $ax \equiv b\pmod m$ has a solution (that is, there exists an $x$ that satisfies this equation) if and only if $b\equiv0\pmod d$. (Hint: If $b \equiv 0 \pmod d$, we can get a useful equation by dividing the equation $ax \equiv b \pmod m$ by $d$.)

	\Part Let $\gcd(m, a) = d$.
	Assume $b \equiv 0\pmod d$.
	Prove that $ax \equiv b\pmod m$ has exactly $d$ solutions (modulo $m$).

    \Part Solve for $x$: $77x \equiv 35 \pmod{42}$.

\end{Parts}
\newpage


\Question{CRT Decomposition}

In this problem we will find $3^{302} \mod{385}$.

\begin{Parts}

\Part
Write $385$ as a product of prime numbers in the form $385=p_1\times p_2 \times p_3$.


\Part
Use Fermat's Little Theorem to find $3^{302} \mod{p_1}$, $3^{302} \mod{p_2}$, and $3^{302}\mod{p_3}$.

\Part
Let $x=3^{302}$. Use part $(b)$ to express the problem as a system of congruences (modular equations $\mod{385}$). Solve the system using the Chinese Remainder Theorem. What is $3^{302}\mod{385}$?

\end{Parts}


\Question{Euler's Totient Function}

  Euler's totient function is defined as follows:
  $$\phi(n) = | \{i: 1 \leq i \leq n, \texttt{gcd}(n,i) = 1\} |$$
  In other words, $\phi(n)$ is the total number of positive integers less than or equal to $n$ which are relatively prime to it.
  Here is a property of Euler's totient function that you can use without proof:

  For $m,n$ such that \texttt{gcd}($m,n$) = 1, $\phi(mn) = \phi(m) \cdot \phi(n)$.

  \begin{Parts}
  \Part
    Let $p$ be a prime number.
    What is $\phi(p)$?

    \Part
    Let $p$ be a prime number and $k$ be some positive integer.
    What is $\phi(p^k)$?


    \Part
    Let $p$ be a prime number and $a$ be a positive integer smaller than $p$.
    What is $a^{\phi(p)} \pmod{p}$?\\\emph{(Hint: use Fermat's Little Theorem.)}

    \Part
    Let $b$ be a number whose prime factors are $p_1,p_2,\hdots, p_k$.
    We can write $b = p_1^{\alpha_1}\cdot p_2^{\alpha_2} \hdots p_k^{\alpha_k}$.

    Show that for any $a$ relatively prime to $b$, the following holds:
    $$\forall i \in \{1,2,\hdots,k\}, \hspace{0.2cm} a^{\phi(b)} \equiv 1 \pmod{p_i}$$

  \end{Parts}


\Question{FLT Converse}

Recall that the FLT states that, given a prime $n$, $a^{n-1} \equiv 1 \pmod{n}$ \textit{for all $1 \leq a \leq n-1$}. Note that it says nothing about when $n$ is composite.

Can the FLT condition ($a^{n-1} \equiv 1 \mod n$) hold for some or even all $a$ if $n$ is composite? This problem will investigate both possibilities. Unlike in the prime case, we need to restrict ourselves to looking at $a$ that are relatively prime to $n$. Because of this restriction, let's define $$S(n) =  \{i: 1 \leq i \leq n, \texttt{gcd}(n,i) = 1\},$$ so $\mid S \mid$ is the total number of possible choices for $a$.
\begin{Parts}
    \Part First, let's show the FLT condition breaks for most choices of $a$ and $n$. More precisely, show that if we can find a single $a \in S(n)$ such that $a^{n-1} \not \equiv 1 \pmod{n}$, we can find at least $|S(n)|/2$ more such $a$. (Hint: Find a bijection that helps you bound the number of values that pass the FLT condition, and remember we only care about values in the set $S$)

\end{Parts}
The above tells us that if a composite number fails the FLT condition for even one number relatively prime to it, then it fails the condition for most numbers relatively prime to it. However, it doesn't rule out the possibility that some composite number $n$ satisifes the FLT condition entirely: \emph{for all} $a$ relatively prime to $n$, $a^{n-1}\equiv 1 \mod n$. It turns out such numbers do exist, but they were found through trial-and-error! We will prove one of the conditions on $n$ that make it easy to verify the existence of these numbers.
\begin{ResumeParts}
    \Part First, show that if $a \equiv b \mod m_1$ and $a \equiv b \mod m_2$, with $\gcd(m_1, m_2)=1$, then $a \equiv b \mod m_1 m_2$.
    
    \Part Let $n = p_1 p_2 \cdots p_k$ where $p_i$ are primes and $p_i - 1 \mid n - 1$ for all $i$. Show that $a^{n-1} \equiv 1 \pmod{n}$ for all $a \in S(n)$.

    \Part Verify that for all $a$ coprime with 561, $a^{560} \equiv 1 \mod 561$.
\end{ResumeParts}


